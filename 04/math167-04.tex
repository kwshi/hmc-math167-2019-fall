\documentclass{../math167}

\title{Homework 4}
\author{}
\date{26 September 2019}

\begin{document}
\begin{problems}
\item Problem 8.4.2 of the text.

  \begin{book}
    A \emph{linear-time reduction} \(R\) must complete its output
    \(R(x)\) in \(\O(\abs x)\) steps.  Prove that there are no
    \(\P\)-complete problems under linear-time reductions.  (Such a
    problem would be in \(\TIME(n^k)\) for some fixed \(k>0\).)
  \end{book}

  \begin{solution}
  \end{solution}

\item Part (a) of Problem 8.4.7 of the text.  (Your circuits should
  use only \(\AND\)-, \(\OR\)-, and \(\NOT\)-gates.  Don't worry if
  you can't solve Part (b); I believe it is misstated---in any case it
  is not solved in the paper by Dymond and Cook that is cited.)

  \begin{book}
    \begin{problems}
    \item Prove that \(\mathrm{CIRCUIT\,VALUE}\) remains
      \(P\)-complete even if the circuit is planar.  (Show how wires
      can cross with no harm to the computed value.)
    \end{problems}
  \end{book}

  \begin{solution}
  \end{solution}

\item Prove that \(\mathrm{CIRCUIT\,VALUE}\) remains \(\P\)-complete
  even if the circuit is monotone (that is, contains only \(\AND\)-
  and \(\OR\)-gates).  (Hint: recall the solution to Problem 1 of
  Homework Assignment 2.)

  \begin{solution}
  \end{solution}

\end{problems}
\end{document}