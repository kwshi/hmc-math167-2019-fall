\documentclass{../math167}

\title{Homework 2}
\date{September 19}
\author{}

\newcommand{\NOT}{\operatorname{NOT}}
\newcommand{\AND}{\operatorname{AND}}
\newcommand{\OR}{\operatorname{OR}}

\begin{document}
\begin{problems}
\item Suppose that a Boolean circuit with \(n\) inputs has \(a\)
  \(\AND\)- and \(\OR\)-gates and \(b\) \(\NOT\)-gates.  Show that the
  same Boolean function can be computed by a circuit with \(2a\)
  \(\AND\)- and \(\OR\)-gates and \(n\) \(\NOT\)-gates.

  \begin{solution}
  \end{solution}

\item Construct a Boolean circuit that has three inputs \(x\), \(y\),
  and \(z\), and three outputs \(\NOT(x)\), \(\NOT(y)\), and
  \(\NOT(z)\).  You may use any number of \(\AND\)- and \(\OR\)-gates
  but only \emph{two} \(\NOT\)-gates.

  \begin{solution}
  \end{solution}

\item Problem 4.4.13 of the text.
  \begin{book}
    A \emph{monotone} Boolean function \(F\) is one that has the
    following property: If one of the inputs changes from
    \(\mathbf{false}\) to \(\mathbf{true}\), the value of the function
    cannot change from \(\mathbf{true}\) to \(\mathbf{false}\).  Show
    that \(F\) is monotone if and only if it can be expressed as a
    circuit with only \(\AND\) and \(\OR\) gates.
  \end{book}

  \begin{solution}
  \end{solution}

\end{problems}
\end{document}