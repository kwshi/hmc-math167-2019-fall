\documentclass{../math167}

\title{Homework 3}
\author{}
\date{September 26}

\begin{document}
\begin{problems}
\item Problem 7.4.4 of the text.
  \begin{book}
    Let \(C\) be a \emph{class of functions} from nonnegative integers
    to nonnegative integers.  We say that \(C\) is closed under
    \emph{left polynomial composition} of \(f(n) \in C\) implies
    \(p(f(n)) = \O(g(n))\) for some \(g(n) \in C\), for all
    polynomials \(p(n)\).  We say that \(C\) is closed under
    \emph{right polynomial composition} if \(f(n) \in C\) implies
    \(f(p(n)) = \O(g(n))\) for some \(g(n) \in C\), for all
    polynomials \(p(n)\).

    Intuitively, the first closure property implies that the
    corresponding complexity class is ``computational
    model-independent'', that is, it is robust under reasonable
    changes in the underlying model of computation (from RAM's to
    Turing machines, to multistring Turing machines, etc.) while
    closure under right polynomial composition suggests closure under
    \emph{reductions} (see the next chapter).

    Which of the following classes of functions are closed under left
    polynomial composition, and which under right polynomial
    composition?
    \begin{problems}
    \item \(\set{n^k : k > 0}\).
    \item \(\set{k \cdot n : k > 0}\).
    \item \(\set{k^n : k > 0}\).
    \item \(\set{2^{n^k} : k > 0}\).
    \item \(\set{\log^k n : k > 0}\).
    \item \(\set{\log n : k > 0}\).
    \end{problems}
  \end{book}

  \begin{solution}
    \begin{problems}
    \item
    \item
    \item
    \item
    \item
    \item
    \end{problems}
  \end{solution}

\item Problem 7.4.6 of the text.
  \begin{book}
    Define the \emph{Kleene star} of a language \(L\) to be
    \(L^* = \set{x_1 \dots x_k : k \ge 0; x_1, \dots, x_k \in L}\)
    notice that our notation \(\Sigma*\) is compatible with this
    definition).  Show that \(\NP\) is closed under Kleene star.
    Repeat for \(\P\).  (This last one is a little less obvious.)
  \end{book}

  \begin{solution}
  \end{solution}

\item Problem 7.4.7 of the text.
  \begin{book}
    Show that \(\NP \ne \SPACE(n)\).  (We have no idea if one includes
    the other, but we know they are different!  Obviously, closure
    under some operation must be used.)
  \end{book}

  \begin{solution}
  \end{solution}

\end{problems}
\end{document}