\documentclass{../math167}

\title{Homework 8}
\author{}
\date{15 November}

\begin{document}
\begin{problems}
\item A \emph{tally language} is a language over a one-letter alphabet
  (that is, a subset of \(\set 0^*\)).  Show that \(L \in \P^A\) for
  some sparse oracle \(A\) if and only if \(L \in \P^B\) for some
  tally language \(B\).

  \begin{solution}
  \end{solution}

\item \newcommand{\TI}{\problem{TI}} A \emph{tree} is a connected and
  acyclic undirected graph.  Define the \emph{tree isomorphism}
  problem \(\TI = \set{\abr{T, T'} : \text{\(T\) and \(T'\) are
      isomorphic trees}}\).  Show that \(\TI \in \P\).

  \begin{solution}
  \end{solution}

\item \newcommand{\FVP}{\problem{FVP}} Consider fully-parenthesized
  Boolean formulas with logical connectives \(\lnot\), \(\land\),
  \(\lor\), and \(\oplus\) (\(\NOT\), \(\AND\), \(\OR\), and
  \(\mathrm{EXCLUSIVE\,OR}\), for example
  \(((x_1 \lor x) \land (x_1 \oplus (\lnot x_2)))\).  Define
  \[
    \FVP = \set{\abr{\Phi(x_1, \dots, x_m), c_1, \dots, c_m}
      : \text{\(\Phi(x_1, \dots, x_m)\)
        is a fully-parenthesized formula,
        and \(\Phi(c_1, \dots, c_m) = 1\)}}.
  \]
  Show that \(\FVP \in \L\).

  \begin{solution}
  \end{solution}

\end{problems}
\end{document}