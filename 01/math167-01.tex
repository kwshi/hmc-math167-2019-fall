\documentclass{../math167}

\title{Homework 1}
\author{}
\date{September 12}

\begin{document}

\begin{problems}
\item Problem 2.8.7(b) of the text.
  \begin{book}
    Suppose that we have a Turing machine with an infinite
    \emph{two-dimensional string} (blackboard?). There are now moves
    of the form \(\uparrow\) and \(\downarrow\), along with
    \(\leftarrow\) and \(\rightarrow\).  The input is written
    initially to the right of the initial cursor position.
    \begin{problems}[start=2]
    \item Show that such a machine can be simulated by a 3-string
      Turing machine with a \emph{quadratic} loss of efficiency.
    \end{problems}
  \end{book}
  (Here, ``\emph{quadratic} loss of efficiency'' means that what the
  simulated machine does in the first \(s\) steps, the simulating
  machine accomplishes in its first \(O(s^2)\) steps.)

  \begin{solution}
  \end{solution}

\item Suppose that we have a Turing machine with a single string, but
  with two independently movable cursors on it.  (Each cursor can read
  the symbol of the string at which it is positioned, whether that
  symbol was written by it or by the other cursor.)  Show that such a
  machine can be simulated by a 2-string Turing machine (with each
  string having its own cursor, as usual) with a \emph{linear} loss of
  efficiency.  (Here, ``\emph{linear} loss of efficiency'' means that
  what the simulated machine does it its first \(s\) steps, the
  simulating machine accomplishes in its first \(O(s)\) steps.)

  \begin{solution}
  \end{solution}

\item Problem 2.8.17 of the text.
  \begin{book}
    Show that any language decided by a \(k\)-string nondeterministic
    Turing machine within time \(f(n)\) can be decided by a 2-string
    nondeterministic Turing machine \emph{also} within time \(f(n)\).
    (Discovering the simple solution is an excellent exercise for
    understanding nondeterminism.  Compare with Theorem 2.1 and
    Problem 2.8.9 for deterministic machines.)

    Needless to say, any \(k\)-string nondeterministic Turing machine
    can be simulated by a single-string nondeterministic machine with
    a quadratic loss of efficiency, exactly as with deterministic
    machines.
  \end{book}

  \begin{solution}
  \end{solution}
\end{problems}

\end{document}