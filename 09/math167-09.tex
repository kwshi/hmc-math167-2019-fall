\documentclass{../math167}

\title{Homework 9}
\author{}
\date{21 November}

\begin{document}
\begin{problems}
\item Problem 14.5.10 of the text.  (Hint: Do part (b) first.)

  \newcommand{\h}{_\mathrm h}

  \begin{book}

    Define a \emph{robust} oracle machine \(M^?\) deciding language
    \(L\) to be one such that \(L(M^A) = L\) for all oracles \(A\).
    That is, the answers are always correct, independently of the
    oracle (although the number of steps may vary from oracle to
    oracle).  If furthermore \(M^A\) works in polynomial time, we say
    that oracle \(A\) \emph{helps} the robust machine \(M^?\).  Let
    \(\P\h\) be the class of languages decidable in polynomial time by
    deterministic robust oracle machines that can be helped; and
    \(\NP\h\) for nondeterministic machines.
    \begin{problems}
    \item Show that \(\P\h = \NP \cap \co\NP\).
    \item Show that \(\NP\h = \NP\).
    \end{problems}
  \end{book}

  \begin{solution}
    \begin{problems}
    \item
    \item
    \end{problems}
  \end{solution}

\item Reduce the computation of the parity
  \(x_1 \oplus \dots \oplus x_n\) of \(n\) Boolean variables to the
  multiplication of two \(n^2\)-bit factors (represented in binary) to
  form a \(2n^2\)-bit product, by setting some of the bits of the
  factors to constants in such a way that the parity of the remaining
  bits emerges as one of the bits of the product.

  \begin{solution}
  \end{solution}

\item Show how the \(n\) Boolean functions
  \(y_1=x_1, y_2 = x_1 \lor x_2, \dots, y_n = x_1 \lor x_2 \lor \cdots
  \lor x_n\) of the \(n\) Boolean variables \(x_1, x_2, \dots, x_n\)
  can be computed by an unbounded fan-in circuit of \(\NOT\)-,
  \(\AND\)- and \(\OR\)-gates of size \(\O(n)\) and depth
  \(\O(1)\). (Hint: Partition the \(n\) inputs into about \(\sqrt n\)
  blocks of about \(\sqrt n\) inputs each.)

  \begin{solution}
  \end{solution}

\end{problems}
\end{document}