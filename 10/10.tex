\documentclass{../math167}

\title{Homework 10}
\author{}
\date{5 December 2019}

\begin{document}
\begin{problems}
\item Consider the tree underlying a logical formula; that is, a tree
  in which every internal vertex has degree one or two; and regard
  each edge as being directed toward the root.  Recall that the
  \emph{pebble number} of such a tree is the smallest number of
  pebbles sufficient to allow a pebble to be placed on the root, with
  the following rules:
  \begin{enumerate*}
  \item a pebble may be placed on a vertex provided all immediate
    predecessors of that vertex currently have pebbles on them, and
  \item a pebble may be removed from a vertex at any time.
  \end{enumerate*}
  Show that there is a polynomial-time algorithm for determining the
  pebble number of such a tree.

  \begin{solution}
  \end{solution}

\item Show that in the unbounded-fan-in model for circuits (gates with
  fan-in \(m\) contribute \(m\) to the size, but only \(1\) to the
  depth), the parity function
  \(f(x_1, \dots, x_n) = x_1 \oplus \dots \oplus x_n\) can be computed
  by circuits with polynomial size and depth
  \(\O(\log n / \log \log n)\).

  \begin{solution}
  \end{solution}

\item Show that the majority function
  \[
    f(x_1, \dots, x_{2m+1}) =
    \begin{cases}
      1, & \text{if at least \(m+1\) of the inputs are \(1\)s}, \\
      0, & \text{otherwise},
    \end{cases}
  \]
  can be computed by circuits (ordinary circuits, with fan-in two) of
  depth \(\O\prn*{\log(2m+1)}\).  (Hint: Let \(2m+1 = 2^k-1\); divide
  and conquer!)

  \begin{solution}
  \end{solution}

\end{problems}
\end{document}