\documentclass{../math167}

\title{Homework 7}
\author{}
\date{7 November}

\begin{document}
\begin{problems}
\item Problem 11.5.17 of the text.

  \begin{book}
    \newcommand{\PPe}{\PP_\mathbfit\epsilon}

    Let \(0 < \epsilon < 1\) be a rational number.  We say that
    \(L \in \PPe\) if there is a nondeterministic Turing machine \(M\)
    such that \(x \in L\) if and only if at least an \(\epsilon\)
    fraction of the computations are accepting.  Show that
    \(\PPe = \PP\).
  \end{book}

  \begin{solution}
  \end{solution}

\item Problem 11.5.18 of the text.

  \begin{book}
    Show that, if \(\NP \subseteq \BPP\), then \(\RP = \NP\).  (That
    is, if \SAT{} can be solved by randomized machines, then it can be
    solved by randomized machines with no false positvies, presumably
    by computing a satisfying truth assignment as in Example 10.3.)
  \end{book}

  \begin{solution}
  \end{solution}

\item Problem 11.5.25 of the text.

  \begin{book}
    We know that most languages do not have polynomial circuits
    (Theorem 4.3), but that certain undecidable ones do (Proposition
    11.2).  We suspect that \(\NP\)-complete languages have no
    polynomial circuits (Conjecture B in Section 11.4).  How high do
    we have to go in complexity to find languages that \emph{provably}
    do not have polynomial circuits?
  \end{book}

  \begin{solution}
  \end{solution}

\end{problems}
\end{document}